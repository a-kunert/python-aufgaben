\documentclass[a4paper]{article} 
\usepackage{minted}
\usemintedstyle{friendly}
\addtolength{\hoffset}{-2.25cm}
\addtolength{\textwidth}{4.5cm}
\addtolength{\voffset}{-3.25cm}
\addtolength{\textheight}{5cm}
\setlength{\parskip}{0pt}
\setlength{\parindent}{0in}

%----------------------------------------------------------------------------------------
%	PACKAGES AND OTHER DOCUMENT CONFIGURATIONS
%----------------------------------------------------------------------------------------

\usepackage{blindtext} % Package to generate dummy text
\usepackage{charter} % Use the Charter font
%\usepackage{lmodern} % Use the Charter font
\usepackage[utf8]{inputenc} % Use UTF-8 encoding
\usepackage{microtype} % Slightly tweak font spacing for aesthetics
\usepackage[english, ngerman]{babel} % Language hyphenation and typographical rules
\usepackage{amsthm, amsmath, amssymb} % Mathematical typesetting
\usepackage{float} % Improved interface for floating objects
\usepackage[final, colorlinks = true, 
            linkcolor = black, 
            citecolor = black]{hyperref} % For hyperlinks in the PDF
\usepackage{graphicx, multicol} % Enhanced support for graphics
\usepackage{xcolor} % Driver-independent color extensions
\usepackage{marvosym, wasysym} % More symbols
\usepackage{rotating} % Rotation tools
\usepackage{censor} % Facilities for controlling restricted text
\usepackage{listings, style/lstlisting} % Environment for non-formatted code, !uses style file!
\usepackage{pseudocode} % Environment for specifying algorithms in a natural way
\usepackage{style/avm} % Environment for f-structures, !uses style file!
\usepackage{booktabs} % Enhances quality of tables
\usepackage{tikz-qtree} % Easy tree drawing tool
\usepackage{ifthen}
\usepackage{lastpage}
\usepackage{titlesec}
\tikzset{every tree node/.style={align=center,anchor=north},
         level distance=2cm} % Configuration for q-trees
\usepackage{style/btree} % Configuration for b-trees and b+-trees, !uses style file!
\usepackage[backend=biber,style=numeric,
            sorting=nyt]{biblatex} % Complete reimplementation of bibliographic facilities
\addbibresource{ecl.bib}
\usepackage{csquotes} % Context sensitive quotation facilities
\usepackage{fancyhdr} % Headers and footers
\pagestyle{fancy} % All pages have headers and footers
\fancyhead{}\renewcommand{\headrulewidth}{0pt} % Blank out the default header
\fancyfoot[L]{\thesheetsubmission{}} % Custom footer text
\fancyfoot[C]{} % Custom footer text
\fancyfoot[R]{\ifthenelse{\pageref{LastPage} > 1}{\footnotesize Seite \thepage{} von \pageref{LastPage} }} % Custom footer text
\newcommand{\note}[1]{\marginpar{\scriptsize \textcolor{red}{#1}}} % Enables comments in red on margin

\titleformat{\section}
{\normalfont\Large\bfseries}{Aufgabe~\thesection}{1em}{\normalsize}
\titlespacing*{\section}{0em}{6ex}{2ex}


\renewcommand{\theenumi}{\alph{enumi}}
\renewcommand\labelenumi{(\theenumi)}

\newcommand*{\sheetdate}[1]{\def\thesheetdate{#1}}
\newcommand*{\sheetsubmission}[1]{\def\thesheetsubmission{#1}}
\newcommand*{\sheetnumber}[1]{\def\thesheetnumber{#1}}


%----------------------------------------------------------------------------------------


\sheetsubmission{}
\sheetnumber{4}
\sheetdate{20.5.2021}


\begin{document}

%-------------------------------
%	TITLE SECTION
%-------------------------------

\fancyhead[C]{}
\hrule \medskip % Upper rule
\begin{minipage}[t]{0.295\textwidth}
\raggedright
\footnotesize
Dr. Aaron Kunert \hfill\\   
aaron.kunert@salemkolleg.de \hfill \\
\end{minipage}
\begin{minipage}[t]{0.4\textwidth} 
\centering 
\large 
Einführung in Python\\ 
\normalsize 
Lösung zu Blatt \thesheetnumber{}\\ 
\end{minipage}
\begin{minipage}[t]{0.295\textwidth} 
\raggedleft
\footnotesize
\thesheetdate{}
\hfill\\
\end{minipage}
\medskip\hrule 
\bigskip

%-------------------------------
%	CONTENTS
%-------------------------------

\section{}
\begin{minted}{python}
def merge_dict(first_dict, second_dict):
  result = {}
  for key, value in first_dict.items():
    result[key] = value
  for key, value in second_dict.items():
    result[key] = value
  return result
\end{minted}

\section{}
\begin{minted}{python}
def find_max(dictionary):
  # Finde das Maximum
  values = list(dictionary.values())
  # Bevor man mehr weiß,
  # ist der erste Wert das mögliche Maximum
  maximum = values[0]
  for value in values:
    # Ist ein Wert größer als das bisher größte
    # gefunde Element, wird dies das neue Maximum.
    if value > maximum:
      maximum = value
  # Wenn man das Maximum kennt,
  # finde alle Schlüssel, wo das Maximum
  # angenommen wird.
  result = []
  for key, value in dictionary.items():
    if value == maximum:
      result.append(key)
  return result
\end{minted}

\section{} 
\begin{minted}{python}
def my_func(string):
  result = ""
  for k in range(1, len(string) + 1):
    result = result + string[-k]
  return result

# Alternative und sehr elegante Lösung
def my_func(string):
  return string[::-1]
\end{minted}
\newpage
\section{}
\begin{minted}{python}
# Nutze die Funktion aus der letzten Aufgabe
def reverse_string(string):
  return string[::-1]


def is_palindrome(string):
  # Normiere die Eingabe, indem sie in Kleinbuchstaben
  # umgewandelt wird
  lowercase_string = string.lower()
  # Ein String ist genau dann ein Palindrom, wenn er mit
  # auch nach dem umdrehen mit sich übereinstimm
  return reverse_string(lowercase_string) == lowercase_string
\end{minted}

\end{document}
