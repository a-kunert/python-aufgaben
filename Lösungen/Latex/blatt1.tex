\documentclass[a4paper]{article} 
\input{head}

\sheetsubmission{Abgabe bis Di. 20.4.2021 per E-Mail}
\sheetnumber{1}
\sheetdate{15.4.2021}


\begin{document}

%-------------------------------
%	TITLE SECTION
%-------------------------------

\input{header}

%-------------------------------
%	CONTENTS
%-------------------------------

\section{}
Schreibe ein kleines Programm, welches folgenden String exakt wie angegeben ausgibt:\\
\texttt{In Python können Strings mittels ' oder "{} oder "{}"{}"{} definiert werden.}
\section{}
Schreibe ein Programm, dass Deinen Namen, Straße, Hausnummer, Postleitzahl und Wohnort abfragt und in der folgenden Formatierung auf der Konsole ausgibt:\\ \\
\texttt{Vorname Nachname}\\
\texttt{Straße Nummer}\\
\texttt{Postleitzahl Wohnort}\\ \\
Versuche dabei nur einziges mal die \texttt{print}-Funktion zu verwenden.  
\section{Zinsrechner} 
Schreibe wie folgt einen einfachen Zinsrechner: Am ersten Tag eines jeden Jahres wird eine Sparrate eingezahlt. Am Ende eines jeden Jahres werden 3\% Zinsen auf den Kontostand gutgeschrieben. 
Das Programm soll die Werte für den \textit{anfänglichen Kontostand} und eine \textit{jährliche Sparrate} vom Benutzer erfragen. Daraufhin sollen die Kontostände der kommenden vier Jahre auf der Konsole ausgegeben werden.     

\section{Einheiten umrechnen}
Schreibe wie folgt jeweils einige Einheitenkonverter. Diese sollen mittels \texttt{input} eine Zahl einlesen und das Ergebnis der Umrechnung auf der Konsole ausgeben. 
\begin{enumerate}
\item Umrechnung von Meter in Kilometer.
\item Umrechnung von Grad Celsius in Grad Kelvin.
\item Umrechnung von Zentimeter in Zoll.
\item Umrechnung von Grad Celsius in Grad Fahrenheit.
\item Umrechnung von Metern in Fuß und Zoll. Die Ausgabe bei der Eingabe $2$ sollte etwa wie folgt aussehen: \texttt{2 Meter entspricht 6ft 7in}. 
\end{enumerate}

\end{document}
