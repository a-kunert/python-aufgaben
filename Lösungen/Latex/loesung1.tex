\documentclass[a4paper]{article} 
\usepackage{minted}
\usemintedstyle{friendly}
\input{head}

\sheetsubmission{}
\sheetnumber{1}
\sheetdate{22.4.2021}


\begin{document}

%-------------------------------
%	TITLE SECTION
%-------------------------------

\input{header}

%-------------------------------
%	CONTENTS
%-------------------------------

\section{}
\begin{minted}{python}
# double quotes have to be escaped with a backslash
print("In Pyhton können Strings mittels ' oder \" oder \"\"\" definiert werden.")
\end{minted}

\section{}
\begin{minted}{python}
first_name = input("Dein Vorname: ")
last_name = input("Dein Nachname: ")
street = input("Deine Straße: ")
house_number = input("Deine Hausnummer: ")
zip_code = input("Deine PLZ: ")
city = input("Deine Stadt: ")

# linefeeds can be created by using the special charakter \n
print(f"{first_name} {last_name}\n{street} {house_number}\n{zip_code} {city}")
\end{minted}


\section{} 
\begin{minted}{python}
initial_balance = input("Anfänglicher Kontostand: ")
savings_rate = input("Sparrate: ")

# cast strings to floats
initial_balance = float(initial_balance)
savings_rate = float(savings_rate)

# transform everything to integer cents instead of float euros
initial_balance = int(initial_balance * 100)
savings_rate = int(savings_rate * 100)

balance = initial_balance

# after 1st year
balance += savings_rate  # savings rate is added at start of year
interest = 0.03 * balance
balance += int(interest)
# Alternatively you can also compute the new balance by: balance = int(1.03 * balance)
# division by 100 gives results in eur
print(f"Der Kontostand nach einem Jahr beträgt: {balance/100}")

# after 2nd year
balance += savings_rate
balance *= 1.03
balance = int(balance)
print(f"Der Kontostand nach zwei Jahren beträgt: {balance/100}")

# after 3rd year
balance += savings_rate
balance *= 1.03
balance = int(balance)
print(f"Der Kontostand nach drei Jahren beträgt: {balance/100}")
\end{minted}
\newpage
\begin{minted}{python}
# after 4th year
balance += savings_rate
balance *= 1.03
balance = int(balance)
print(f"Der Kontostand nach vier Jahren beträgt: {balance/100}")
\end{minted}

\section{}
\begin{minted}{python}
# Part (a)
length = input("Länge in Meter: ")
length = float(length)
print(f"Die Länge in Kilometern beträgt {length/1000}km")

# Part (b)
temp = input("Temperatur in Grad Celsius: ")
temp = float(temp)
temp = temp + 273.15
temp = int(temp * 100)/100  # my creative way to round to 2 figures
print(f"Die Temperatur in Grad Kelvin beträgt {temp}")

# Part (c)
length = input("Länge in Zentimeter: ")
length = float(length)
length = length/2.54
length = int(length*100)/100  # my creative way to round to 2 figures
print(f"Die Länge in Zoll beträgt {length}")

# Part (d)
temp = input("Temperatur in Grad Celsius: ")
temp = float(temp)
temp = 9/5 * temp + 32  # you find this conversion rule on Wikipedia
temp = int(temp * 100)/100  # my creative way to round to 2 figures
print(f"Die Temperatur in Grad Fahrenheit beträgt {temp}")


# Part (e)
FOOT_IN_METER = 0.3048  # By convention uppercase names are used for constants
INCH_IN_METER = 0.0254
length = input("Länge in Metern: ")
length = float(length)
# compute the whole feet contained in the length using integer division:
feet = length // FOOT_IN_METER  
feet = int(feet)
# compute the remainder after deducting an integer number of feet
remainder = length % FOOT_IN_METER  
inches = remainder / INCH_IN_METER  # convert the remainder to inches
inches = int(inches * 100)/100  # round creatively
print(f'{length} Meter entsprechedn {feet} ft. {inches}"')
\end{minted}

\end{document}
