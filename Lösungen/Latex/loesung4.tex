\documentclass[a4paper]{article} 
\usepackage{minted}
\usemintedstyle{friendly}
\input{head}

\sheetsubmission{}
\sheetnumber{4}
\sheetdate{20.5.2021}


\begin{document}

%-------------------------------
%	TITLE SECTION
%-------------------------------

\input{header}

%-------------------------------
%	CONTENTS
%-------------------------------

\section{}
\begin{minted}{python}
def merge_dict(first_dict, second_dict):
  result = {}
  for key, value in first_dict.items():
    result[key] = value
  for key, value in second_dict.items():
    result[key] = value
  return result
\end{minted}

\section{}
\begin{minted}{python}
def find_max(dictionary):
  # Finde das Maximum
  values = list(dictionary.values())
  # Bevor man mehr weiß,
  # ist der erste Wert das mögliche Maximum
  maximum = values[0]
  for value in values:
    # Ist ein Wert größer als das bisher größte
    # gefunde Element, wird dies das neue Maximum.
    if value > maximum:
      maximum = value
  # Wenn man das Maximum kennt,
  # finde alle Schlüssel, wo das Maximum
  # angenommen wird.
  result = []
  for key, value in dictionary.items():
    if value == maximum:
      result.append(key)
  return result
\end{minted}

\section{} 
\begin{minted}{python}
def my_func(string):
  result = ""
  for k in range(1, len(string) + 1):
    result = result + string[-k]
  return result

# Alternative und sehr elegante Lösung
def my_func(string):
  return string[::-1]
\end{minted}
\newpage
\section{}
\begin{minted}{python}
# Nutze die Funktion aus der letzten Aufgabe
def reverse_string(string):
  return string[::-1]


def is_palindrome(string):
  # Normiere die Eingabe, indem sie in Kleinbuchstaben
  # umgewandelt wird
  lowercase_string = string.lower()
  # Ein String ist genau dann ein Palindrom, wenn er mit
  # auch nach dem umdrehen mit sich übereinstimm
  return reverse_string(lowercase_string) == lowercase_string
\end{minted}

\end{document}
