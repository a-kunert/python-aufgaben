\documentclass[a4paper]{article} 
\addtolength{\hoffset}{-2.25cm}
\addtolength{\textwidth}{4.5cm}
\addtolength{\voffset}{-3.25cm}
\addtolength{\textheight}{5cm}
\setlength{\parskip}{0pt}
\setlength{\parindent}{0in}

%----------------------------------------------------------------------------------------
%	PACKAGES AND OTHER DOCUMENT CONFIGURATIONS
%----------------------------------------------------------------------------------------

\usepackage{blindtext} % Package to generate dummy text
\usepackage{charter} % Use the Charter font
%\usepackage{lmodern} % Use the Charter font
\usepackage[utf8]{inputenc} % Use UTF-8 encoding
\usepackage{microtype} % Slightly tweak font spacing for aesthetics
\usepackage[english, ngerman]{babel} % Language hyphenation and typographical rules
\usepackage{amsthm, amsmath, amssymb} % Mathematical typesetting
\usepackage{float} % Improved interface for floating objects
\usepackage[final, colorlinks = true, 
            linkcolor = black, 
            citecolor = black]{hyperref} % For hyperlinks in the PDF
\usepackage{graphicx, multicol} % Enhanced support for graphics
\usepackage{xcolor} % Driver-independent color extensions
\usepackage{marvosym, wasysym} % More symbols
\usepackage{rotating} % Rotation tools
\usepackage{censor} % Facilities for controlling restricted text
\usepackage{listings, style/lstlisting} % Environment for non-formatted code, !uses style file!
\usepackage{pseudocode} % Environment for specifying algorithms in a natural way
\usepackage{style/avm} % Environment for f-structures, !uses style file!
\usepackage{booktabs} % Enhances quality of tables
\usepackage{tikz-qtree} % Easy tree drawing tool
\usepackage{ifthen}
\usepackage{lastpage}
\usepackage{titlesec}
\tikzset{every tree node/.style={align=center,anchor=north},
         level distance=2cm} % Configuration for q-trees
\usepackage{style/btree} % Configuration for b-trees and b+-trees, !uses style file!
\usepackage[backend=biber,style=numeric,
            sorting=nyt]{biblatex} % Complete reimplementation of bibliographic facilities
\addbibresource{ecl.bib}
\usepackage{csquotes} % Context sensitive quotation facilities
\usepackage{fancyhdr} % Headers and footers
\pagestyle{fancy} % All pages have headers and footers
\fancyhead{}\renewcommand{\headrulewidth}{0pt} % Blank out the default header
\fancyfoot[L]{\thesheetsubmission{}} % Custom footer text
\fancyfoot[C]{} % Custom footer text
\fancyfoot[R]{\ifthenelse{\pageref{LastPage} > 1}{\footnotesize Seite \thepage{} von \pageref{LastPage} }} % Custom footer text
\newcommand{\note}[1]{\marginpar{\scriptsize \textcolor{red}{#1}}} % Enables comments in red on margin

\titleformat{\section}
{\normalfont\Large\bfseries}{Aufgabe~\thesection}{1em}{\normalsize}
\titlespacing*{\section}{0em}{6ex}{2ex}


\renewcommand{\theenumi}{\alph{enumi}}
\renewcommand\labelenumi{(\theenumi)}

\newcommand*{\sheetdate}[1]{\def\thesheetdate{#1}}
\newcommand*{\sheetsubmission}[1]{\def\thesheetsubmission{#1}}
\newcommand*{\sheetnumber}[1]{\def\thesheetnumber{#1}}


%----------------------------------------------------------------------------------------


\sheetsubmission{Abgabe bis Di. 20.4.2021 per E-Mail}
\sheetnumber{1}
\sheetdate{15.4.2021}


\begin{document}

%-------------------------------
%	TITLE SECTION
%-------------------------------

\fancyhead[C]{}
\hrule \medskip % Upper rule
\begin{minipage}[t]{0.295\textwidth}
\raggedright
\footnotesize
Dr. Aaron Kunert \hfill\\   
aaron.kunert@salemkolleg.de \hfill \\
\end{minipage}
\begin{minipage}[t]{0.4\textwidth} 
\centering 
\large 
Einführung in Python\\ 
\normalsize 
Lösung zu Blatt \thesheetnumber{}\\ 
\end{minipage}
\begin{minipage}[t]{0.295\textwidth} 
\raggedleft
\footnotesize
\thesheetdate{}
\hfill\\
\end{minipage}
\medskip\hrule 
\bigskip

%-------------------------------
%	CONTENTS
%-------------------------------

\section{}
Schreibe ein kleines Programm, welches folgenden String exakt wie angegeben ausgibt:\\
\texttt{In Python können Strings mittels ' oder "{} oder "{}"{}"{} definiert werden.}
\section{}
Schreibe ein Programm, dass Deinen Namen, Straße, Hausnummer, Postleitzahl und Wohnort abfragt und in der folgenden Formatierung auf der Konsole ausgibt:\\ \\
\texttt{Vorname Nachname}\\
\texttt{Straße Nummer}\\
\texttt{Postleitzahl Wohnort}\\ \\
Versuche dabei nur einziges mal die \texttt{print}-Funktion zu verwenden.  
\section{Zinsrechner} 
Schreibe wie folgt einen einfachen Zinsrechner: Am ersten Tag eines jeden Jahres wird eine Sparrate eingezahlt. Am Ende eines jeden Jahres werden 3\% Zinsen auf den Kontostand gutgeschrieben. 
Das Programm soll die Werte für den \textit{anfänglichen Kontostand} und eine \textit{jährliche Sparrate} vom Benutzer erfragen. Daraufhin sollen die Kontostände der kommenden vier Jahre auf der Konsole ausgegeben werden.     

\section{Einheiten umrechnen}
Schreibe wie folgt jeweils einige Einheitenkonverter. Diese sollen mittels \texttt{input} eine Zahl einlesen und das Ergebnis der Umrechnung auf der Konsole ausgeben. 
\begin{enumerate}
\item Umrechnung von Meter in Kilometer.
\item Umrechnung von Grad Celsius in Grad Kelvin.
\item Umrechnung von Zentimeter in Zoll.
\item Umrechnung von Grad Celsius in Grad Fahrenheit.
\item Umrechnung von Metern in Fuß und Zoll. Die Ausgabe bei der Eingabe $2$ sollte etwa wie folgt aussehen: \texttt{2 Meter entspricht 6ft 7in}. 
\end{enumerate}

\end{document}
