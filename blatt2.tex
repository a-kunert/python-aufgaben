\documentclass[a4paper]{article} 
\addtolength{\hoffset}{-2.25cm}
\addtolength{\textwidth}{4.5cm}
\addtolength{\voffset}{-3.25cm}
\addtolength{\textheight}{5cm}
\setlength{\parskip}{0pt}
\setlength{\parindent}{0in}

%----------------------------------------------------------------------------------------
%	PACKAGES AND OTHER DOCUMENT CONFIGURATIONS
%----------------------------------------------------------------------------------------

\usepackage{blindtext} % Package to generate dummy text
\usepackage{charter} % Use the Charter font
%\usepackage{lmodern} % Use the Charter font
\usepackage[utf8]{inputenc} % Use UTF-8 encoding
\usepackage{microtype} % Slightly tweak font spacing for aesthetics
\usepackage[english, ngerman]{babel} % Language hyphenation and typographical rules
\usepackage{amsthm, amsmath, amssymb} % Mathematical typesetting
\usepackage{float} % Improved interface for floating objects
\usepackage[final, colorlinks = true, 
            linkcolor = black, 
            citecolor = black]{hyperref} % For hyperlinks in the PDF
\usepackage{graphicx, multicol} % Enhanced support for graphics
\usepackage{xcolor} % Driver-independent color extensions
\usepackage{marvosym, wasysym} % More symbols
\usepackage{rotating} % Rotation tools
\usepackage{censor} % Facilities for controlling restricted text
\usepackage{listings, style/lstlisting} % Environment for non-formatted code, !uses style file!
\usepackage{pseudocode} % Environment for specifying algorithms in a natural way
\usepackage{style/avm} % Environment for f-structures, !uses style file!
\usepackage{booktabs} % Enhances quality of tables
\usepackage{tikz-qtree} % Easy tree drawing tool
\usepackage{ifthen}
\usepackage{lastpage}
\usepackage{titlesec}
\tikzset{every tree node/.style={align=center,anchor=north},
         level distance=2cm} % Configuration for q-trees
\usepackage{style/btree} % Configuration for b-trees and b+-trees, !uses style file!
\usepackage[backend=biber,style=numeric,
            sorting=nyt]{biblatex} % Complete reimplementation of bibliographic facilities
\addbibresource{ecl.bib}
\usepackage{csquotes} % Context sensitive quotation facilities
\usepackage{fancyhdr} % Headers and footers
\pagestyle{fancy} % All pages have headers and footers
\fancyhead{}\renewcommand{\headrulewidth}{0pt} % Blank out the default header
\fancyfoot[L]{\thesheetsubmission{}} % Custom footer text
\fancyfoot[C]{} % Custom footer text
\fancyfoot[R]{\ifthenelse{\pageref{LastPage} > 1}{\footnotesize Seite \thepage{} von \pageref{LastPage} }} % Custom footer text
\newcommand{\note}[1]{\marginpar{\scriptsize \textcolor{red}{#1}}} % Enables comments in red on margin

\titleformat{\section}
{\normalfont\Large\bfseries}{Aufgabe~\thesection}{1em}{\normalsize}
\titlespacing*{\section}{0em}{6ex}{2ex}


\renewcommand{\theenumi}{\alph{enumi}}
\renewcommand\labelenumi{(\theenumi)}

\newcommand*{\sheetdate}[1]{\def\thesheetdate{#1}}
\newcommand*{\sheetsubmission}[1]{\def\thesheetsubmission{#1}}
\newcommand*{\sheetnumber}[1]{\def\thesheetnumber{#1}}


%----------------------------------------------------------------------------------------


\sheetsubmission{Abgabe bis Di. 4.5.2021 per E-Mail oder Slack}
\sheetnumber{2}
\sheetdate{29.4.2021}


\begin{document}

%-------------------------------
%	TITLE SECTION
%-------------------------------

\fancyhead[C]{}
\hrule \medskip % Upper rule
\begin{minipage}[t]{0.295\textwidth}
\raggedright
\footnotesize
Dr. Aaron Kunert \hfill\\   
aaron.kunert@salemkolleg.de \hfill \\
\end{minipage}
\begin{minipage}[t]{0.4\textwidth} 
\centering 
\large 
Einführung in Python\\ 
\normalsize 
Lösung zu Blatt \thesheetnumber{}\\ 
\end{minipage}
\begin{minipage}[t]{0.295\textwidth} 
\raggedleft
\footnotesize
\thesheetdate{}
\hfill\\
\end{minipage}
\medskip\hrule 
\bigskip

%-------------------------------
%	CONTENTS
%-------------------------------

\section{Median bestimmen}
Lies drei Zahlen \texttt{x},\texttt{y} und \texttt{z} ein. Gib danach den Median dieser drei Zahlen auf der Konsole aus.

\vspace{2pt}

{\footnotesize \textbf{Hinweis:} Der \emph{Median} einer Zahlenmenge, ist die Zahl, die in der Mitte stehen würde, wenn man die Zahlen der Menge der Größe nach geordnet werden. }


\section{FizzBang}
Eine beliebte Aufgabe in Bewerbungsgesprächen ist, den Abzählreim \emph{FizzBang} zu implementieren. Bei dem Reim geht es darum, von 1 bis 100 zu zählen. Ist eine Zahl durch 3 teilbar wird sie durch das Wort \emph{Fizz} ersetzt, ist sie durch 5 teilbar entsprechend durch das Wort \emph{Bang}. Ist sie sowohl durch 3 als auch durch 5 teilbar, ersetzt man sie durch das Wort \emph{FizzBang}.

 Schreib ein kleines Programm, dass die Zahlen von 1 bis 100 entsprechend des Abzählreims auf der Konsole ausgibt. Die ersten Zeilen der Ausgabe, sollen wie folgt aussehen: 

\vspace{0.2cm}
\begin{verbatim}
 1
 2
 Fizz
 4
 Bang
 Fizz
 7
\end{verbatim}
%\phantom{**}\vdots
\textbf{Zusatz:} Schaffst Du es, die Aufgabe mit höchstens zwei Conditionals (also Schlüsselwörter aus der  \texttt{if}-\texttt{elif}-\texttt{else}-Familie) zu lösen? 

\section{}
Lies ein beliebiges Wort (bestehend aus Kleinbuchstaben) ein. Gib danach auf der Konsole aus, wie viele Vokale das Wort enthält. 

\section{Weihnachtsbaum}
Schreibe ein kleines Skript, die folgende Ausgabe auf der Konsole ausgibt und somit einen Weihnachtsbaum zeichnet. Die genaue Größe bleibt Dir überlassen, sollte aber an der breitesten Stelle mindestens 15 Sternchen breit sein. 

\vspace{0.2cm}

\begin{verbatim}
     *
    ***
   *****
  *******
 *********
\end{verbatim}
\phantom{*****.}\vdots
\begin{verbatim}
    |||
    |||
    |||
\end{verbatim}




\end{document}
