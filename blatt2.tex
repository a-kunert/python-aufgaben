\documentclass[a4paper]{article} 
\input{head}

\sheetsubmission{Abgabe bis Di. 4.5.2021 per E-Mail oder Slack}
\sheetnumber{2}
\sheetdate{29.4.2021}


\begin{document}

%-------------------------------
%	TITLE SECTION
%-------------------------------

\input{header}

%-------------------------------
%	CONTENTS
%-------------------------------

\section{Median bestimmen}
Lies drei Zahlen \texttt{x},\texttt{y} und \texttt{z} ein. Gib danach den Median dieser drei Zahlen auf der Konsole aus.

\vspace{2pt}

{\footnotesize \textbf{Hinweis:} Der \emph{Median} einer Zahlenmenge, ist die Zahl, die in der Mitte stehen würde, wenn man die Zahlen der Menge der Größe nach geordnet werden. }


\section{FizzBuzz}
Eine beliebte Aufgabe in Bewerbungsgesprächen ist, den Abzählreim \emph{FizzBuzz} zu implementieren. Bei dem Reim geht es darum, von 1 bis 100 zu zählen. Ist eine Zahl durch 3 teilbar wird sie durch das Wort \emph{Fizz} ersetzt, ist sie durch 5 teilbar entsprechend durch das Wort \emph{Buzz}. Ist sie sowohl durch 3 als auch durch 5 teilbar, ersetzt man sie durch das Wort \emph{FizzBuzz}.

 Schreib ein kleines Programm, dass die Zahlen von 1 bis 100 entsprechend des Abzählreims auf der Konsole ausgibt. Die ersten Zeilen der Ausgabe, sollen wie folgt aussehen: 

\vspace{0.2cm}
\begin{verbatim}
 1
 2
 Fizz
 4
 Buzz
 Fizz
 7
\end{verbatim}
%\phantom{**}\vdots
\textbf{Zusatz:} Schaffst Du es, die Aufgabe mit höchstens drei Conditionals (also Schlüsselwörter aus der  \texttt{if}-\texttt{elif}-\texttt{else}-Familie) zu lösen? 

\section{}
Lies ein beliebiges Wort (bestehend aus Kleinbuchstaben) ein. Gib danach auf der Konsole aus, wie viele Vokale das Wort enthält. 

\section{Weihnachtsbaum}
Schreibe ein kleines Skript, die folgende Ausgabe auf der Konsole ausgibt und somit einen Weihnachtsbaum zeichnet. Die genaue Größe bleibt Dir überlassen, sollte aber an der breitesten Stelle mindestens 15 Sternchen breit sein. 

\vspace{0.2cm}

\begin{verbatim}
     *
    ***
   *****
  *******
 *********
\end{verbatim}
\phantom{*****.}\vdots
\begin{verbatim}
    |||
    |||
    |||
\end{verbatim}




\end{document}
