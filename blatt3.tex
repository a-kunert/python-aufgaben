\documentclass[a4paper]{article} 
\input{head}

\sheetsubmission{Abgabe bis Di. 18.5.2021 per E-Mail oder Slack}
\sheetnumber{3}
\sheetdate{6.5.2021}


\begin{document}

%-------------------------------
%	TITLE SECTION
%-------------------------------

\input{header}

%-------------------------------
%	CONTENTS
%-------------------------------

\section{Shortcuts}
Suche Dir Deine 3 Lieblings-Tastenkürzel in Pycharm heraus. Gelobe dann, dass Du bis zum Ende des Python-Kurses für diese Aktionen nicht mehr die Maus sondern nur noch die Keyboard-Shortcuts verwenden wirst. 

\vspace{2pt}

{\footnotesize\textbf{Hinweis:}
Du findest im First-Class unter \texttt{Handouts/Pycharm\_Keyma} die Liste meiner Lieblings-Shortcuts.}


\section{Maximum einer Liste bestimmen}
Schreibe ein kleines Skript, dass aus einer beliebigen Liste von Zahlen, die größte Zahl findet und auf der Konsole ausgibt. Die Liste soll dabei nicht verändert werden.  

 \vspace{2pt}
{\footnotesize\textbf{Hinweis:} Für die Abgabe kannst Du gerne eine selbstgewählte Liste verwenden. Allerdings sollte das Prinzip auf beliebige Listen anwendbar sein.}



\section{}
Schreibe ein kleines Skript, dass aus zwei beliebigen Listen, alle Elemente auf der Konsole ausgibt, die in beiden Listen vorkommen. 
\vspace{2pt}

{\footnotesize\textbf{Hinweis:} Für die Abgabe kannst Du gerne zwei selbstgewählte Liste verwenden. Allerdings sollte das Prinzip auf beliebige Listen anwendbar sein.}
 



\section{Passwort validieren}
Wenn man sich im Internet ein Passwort vergibt, wird bei der Eingabe oft überprüft, ob das Passwort start genug ist. Schreibe ein kleines Skript, dass einen Kandidaten für ein Passwort einliest. Prüfe, ob folgende Kriterien erfüllt sind
\begin{itemize}
	\item Das Passwort soll zwischen 8 und 20 Zeichen lang sein.
	\item Das Passwort soll mindestens eine Ziffer zwischen 0 und 9 enthalten.
	\item Das Passwort soll mindestens ein Sonderzeichen aus der Liste @, !, \&, ?,  ; enthalten. 
	\item Das Passwort darf kein Leerzeichen, Punkt oder Komma enthalten. 
	\item \emph{Zusatz:} Das Passwort soll mindestens einen Groß-und einen Kleinbuchstaben enthalten. 
\end{itemize}
Erfüllt das Passwort alle Kriterien, gib \texttt{Passwort akzeptiert} ansonsten etwa \texttt{Das Passwort ist zu kurz} auf der Konsole aus. 

\vspace{2pt}
{\footnotesize\textbf{Hinweis:}
Hier könnten Listen, das Keyword \texttt{in} und Flags hilfreich sein. }



\end{document}
