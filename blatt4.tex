\documentclass[a4paper]{article} 
\addtolength{\hoffset}{-2.25cm}
\addtolength{\textwidth}{4.5cm}
\addtolength{\voffset}{-3.25cm}
\addtolength{\textheight}{5cm}
\setlength{\parskip}{0pt}
\setlength{\parindent}{0in}

%----------------------------------------------------------------------------------------
%	PACKAGES AND OTHER DOCUMENT CONFIGURATIONS
%----------------------------------------------------------------------------------------

\usepackage{blindtext} % Package to generate dummy text
\usepackage{charter} % Use the Charter font
%\usepackage{lmodern} % Use the Charter font
\usepackage[utf8]{inputenc} % Use UTF-8 encoding
\usepackage{microtype} % Slightly tweak font spacing for aesthetics
\usepackage[english, ngerman]{babel} % Language hyphenation and typographical rules
\usepackage{amsthm, amsmath, amssymb} % Mathematical typesetting
\usepackage{float} % Improved interface for floating objects
\usepackage[final, colorlinks = true, 
            linkcolor = black, 
            citecolor = black]{hyperref} % For hyperlinks in the PDF
\usepackage{graphicx, multicol} % Enhanced support for graphics
\usepackage{xcolor} % Driver-independent color extensions
\usepackage{marvosym, wasysym} % More symbols
\usepackage{rotating} % Rotation tools
\usepackage{censor} % Facilities for controlling restricted text
\usepackage{listings, style/lstlisting} % Environment for non-formatted code, !uses style file!
\usepackage{pseudocode} % Environment for specifying algorithms in a natural way
\usepackage{style/avm} % Environment for f-structures, !uses style file!
\usepackage{booktabs} % Enhances quality of tables
\usepackage{tikz-qtree} % Easy tree drawing tool
\usepackage{ifthen}
\usepackage{lastpage}
\usepackage{titlesec}
\tikzset{every tree node/.style={align=center,anchor=north},
         level distance=2cm} % Configuration for q-trees
\usepackage{style/btree} % Configuration for b-trees and b+-trees, !uses style file!
\usepackage[backend=biber,style=numeric,
            sorting=nyt]{biblatex} % Complete reimplementation of bibliographic facilities
\addbibresource{ecl.bib}
\usepackage{csquotes} % Context sensitive quotation facilities
\usepackage{fancyhdr} % Headers and footers
\pagestyle{fancy} % All pages have headers and footers
\fancyhead{}\renewcommand{\headrulewidth}{0pt} % Blank out the default header
\fancyfoot[L]{\thesheetsubmission{}} % Custom footer text
\fancyfoot[C]{} % Custom footer text
\fancyfoot[R]{\ifthenelse{\pageref{LastPage} > 1}{\footnotesize Seite \thepage{} von \pageref{LastPage} }} % Custom footer text
\newcommand{\note}[1]{\marginpar{\scriptsize \textcolor{red}{#1}}} % Enables comments in red on margin

\titleformat{\section}
{\normalfont\Large\bfseries}{Aufgabe~\thesection}{1em}{\normalsize}
\titlespacing*{\section}{0em}{6ex}{2ex}


\renewcommand{\theenumi}{\alph{enumi}}
\renewcommand\labelenumi{(\theenumi)}

\newcommand*{\sheetdate}[1]{\def\thesheetdate{#1}}
\newcommand*{\sheetsubmission}[1]{\def\thesheetsubmission{#1}}
\newcommand*{\sheetnumber}[1]{\def\thesheetnumber{#1}}


%----------------------------------------------------------------------------------------

\usepackage{minted}
\usemintedstyle{friendly}
\sheetsubmission{Abgabe bis Di. 8.6.2021 per E-Mail oder Slack}
\sheetnumber{4}
\sheetdate{20.5.2021}
\usepackage{csquotes}
\newcommand{\pybw}[1]{\mintinline[style=bw]{python}{#1}}
\newcommand{\py}[1]{\mintinline{python}{#1}}

\begin{document}

%-------------------------------
%	TITLE SECTION
%-------------------------------

\fancyhead[C]{}
\hrule \medskip % Upper rule
\begin{minipage}[t]{0.295\textwidth}
\raggedright
\footnotesize
Dr. Aaron Kunert \hfill\\   
aaron.kunert@salemkolleg.de \hfill \\
\end{minipage}
\begin{minipage}[t]{0.4\textwidth} 
\centering 
\large 
Einführung in Python\\ 
\normalsize 
Lösung zu Blatt \thesheetnumber{}\\ 
\end{minipage}
\begin{minipage}[t]{0.295\textwidth} 
\raggedleft
\footnotesize
\thesheetdate{}
\hfill\\
\end{minipage}
\medskip\hrule 
\bigskip

%-------------------------------
%	CONTENTS
%-------------------------------

\section{Merge-Funktion}
Schreibe eine Funktion, die zwei Dictionaries zu einem verschmilzt. Falls ein Key in beiden Dictionaries vorkommt, so soll der jeweilige Wert des zweiten Parameters genommen werden. 

\vspace{4pt}
\textbf{Beispiel}: \py{merge_dict({"a":1, "b":2}, {"b":3, "c":4}) = {"a":1, "b":3, "c":4}}.  

\section{Maximalstellen finden}
Schreibe eine Funktion, die ein Dictionary mit \pybw{int}-Werten als Eingabe erwartet. Es soll dann eine Liste aller Schlüssel zurückgegeben werden, an denen der Wert maximal ist. 

\vspace{4pt}
\textbf{Beispiel}: \py{find_max({"a":0, "b":3, "c":2, "d":3, "e":-1})} = \py{["b","d"]}  


\section{}
Schreibe eine Funktion, die einen String erwartet und den String in umgekehrter Reihenfolge zurückgibt. 

\vspace{4pt}
\textbf{Beispiel}: \py{my_func("Hallo") = "ollaH"}  


\section{}
Schreibe eine Funktion, die einen String erwartet und \pybw{True} zurückgibt, falls der String ein Palindrom ist, andernfalls soll sie \pybw{False} zurückgeben. Die Groß-/Kleinschreibung soll dafür nicht berücksichtigt werden. 

\vspace{4pt}

{\footnotesize\textbf{Hinweis:} \\
	Ein Palindrom ist ein Wort bzw. ein Text, der vorwärts wie rückwärts gelesen identisch ist. Ein Beispiel dafür ist das Wort \enquote{Lagerregal}. \\
Um die Groß-/Kleinschreibung zu ignorieren, kann man den String vorab mit der Methode \pybw{.lower()} in Kleinbuchstaben verwandeln. 

\end{document}
