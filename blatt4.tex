\documentclass[a4paper]{article} 
\input{head}
\usepackage{minted}
\usemintedstyle{friendly}
\sheetsubmission{Abgabe bis Di. 8.6.2021 per E-Mail oder Slack}
\sheetnumber{4}
\sheetdate{20.5.2021}
\usepackage{csquotes}
\newcommand{\pybw}[1]{\mintinline[style=bw]{python}{#1}}
\newcommand{\py}[1]{\mintinline{python}{#1}}

\begin{document}

%-------------------------------
%	TITLE SECTION
%-------------------------------

\input{header}

%-------------------------------
%	CONTENTS
%-------------------------------

\section{Merge-Funktion}
Schreibe eine Funktion, die zwei Dictionaries zu einem verschmilzt. Falls ein Key in beiden Dictionaries vorkommt, so soll der jeweilige Wert des zweiten Parameters genommen werden. 

\vspace{4pt}
\textbf{Beispiel}: \py{merge_dict({"a":1, "b":2}, {"b":3, "c":4}) = {"a":1, "b":3, "c":4}}.  

\section{Maximalstellen finden}
Schreibe eine Funktion, die ein Dictionary mit \pybw{int}-Werten als Eingabe erwartet. Es soll dann eine Liste aller Schlüssel zurückgegeben werden, an denen der Wert maximal ist. 

\vspace{4pt}
\textbf{Beispiel}: \py{find_max({"a":0, "b":3, "c":2, "d":3, "e":-1})} = \py{["b","d"]}  


\section{}
Schreibe eine Funktion, die einen String erwartet und den String in umgekehrter Reihenfolge zurückgibt. 

\vspace{4pt}
\textbf{Beispiel}: \py{my_func("Hallo") = "ollaH"}  


\section{}
Schreibe eine Funktion, die einen String erwartet und \pybw{True} zurückgibt, falls der String ein Palindrom ist, andernfalls soll sie \pybw{False} zurückgeben. Die Groß-/Kleinschreibung soll dafür nicht berücksichtigt werden. 

\vspace{4pt}

{\footnotesize\textbf{Hinweis:} \\
	Ein Palindrom ist ein Wort bzw. ein Text, der vorwärts wie rückwärts gelesen identisch ist. Ein Beispiel dafür ist das Wort \enquote{Lagerregal}. \\
Um die Groß-/Kleinschreibung zu ignorieren, kann man den String vorab mit der Methode \pybw{.lower()} in Kleinbuchstaben verwandeln. 

\end{document}
