\documentclass[a4paper]{article} 
\addtolength{\hoffset}{-2.25cm}
\addtolength{\textwidth}{4.5cm}
\addtolength{\voffset}{-3.25cm}
\addtolength{\textheight}{5cm}
\setlength{\parskip}{0pt}
\setlength{\parindent}{0in}

%----------------------------------------------------------------------------------------
%	PACKAGES AND OTHER DOCUMENT CONFIGURATIONS
%----------------------------------------------------------------------------------------

\usepackage{blindtext} % Package to generate dummy text
\usepackage{charter} % Use the Charter font
%\usepackage{lmodern} % Use the Charter font
\usepackage[utf8]{inputenc} % Use UTF-8 encoding
\usepackage{microtype} % Slightly tweak font spacing for aesthetics
\usepackage[english, ngerman]{babel} % Language hyphenation and typographical rules
\usepackage{amsthm, amsmath, amssymb} % Mathematical typesetting
\usepackage{float} % Improved interface for floating objects
\usepackage[final, colorlinks = true, 
            linkcolor = black, 
            citecolor = black]{hyperref} % For hyperlinks in the PDF
\usepackage{graphicx, multicol} % Enhanced support for graphics
\usepackage{xcolor} % Driver-independent color extensions
\usepackage{marvosym, wasysym} % More symbols
\usepackage{rotating} % Rotation tools
\usepackage{censor} % Facilities for controlling restricted text
\usepackage{listings, style/lstlisting} % Environment for non-formatted code, !uses style file!
\usepackage{pseudocode} % Environment for specifying algorithms in a natural way
\usepackage{style/avm} % Environment for f-structures, !uses style file!
\usepackage{booktabs} % Enhances quality of tables
\usepackage{tikz-qtree} % Easy tree drawing tool
\usepackage{ifthen}
\usepackage{lastpage}
\usepackage{titlesec}
\tikzset{every tree node/.style={align=center,anchor=north},
         level distance=2cm} % Configuration for q-trees
\usepackage{style/btree} % Configuration for b-trees and b+-trees, !uses style file!
\usepackage[backend=biber,style=numeric,
            sorting=nyt]{biblatex} % Complete reimplementation of bibliographic facilities
\addbibresource{ecl.bib}
\usepackage{csquotes} % Context sensitive quotation facilities
\usepackage{fancyhdr} % Headers and footers
\pagestyle{fancy} % All pages have headers and footers
\fancyhead{}\renewcommand{\headrulewidth}{0pt} % Blank out the default header
\fancyfoot[L]{\thesheetsubmission{}} % Custom footer text
\fancyfoot[C]{} % Custom footer text
\fancyfoot[R]{\ifthenelse{\pageref{LastPage} > 1}{\footnotesize Seite \thepage{} von \pageref{LastPage} }} % Custom footer text
\newcommand{\note}[1]{\marginpar{\scriptsize \textcolor{red}{#1}}} % Enables comments in red on margin

\titleformat{\section}
{\normalfont\Large\bfseries}{Aufgabe~\thesection}{1em}{\normalsize}
\titlespacing*{\section}{0em}{6ex}{2ex}


\renewcommand{\theenumi}{\alph{enumi}}
\renewcommand\labelenumi{(\theenumi)}

\newcommand*{\sheetdate}[1]{\def\thesheetdate{#1}}
\newcommand*{\sheetsubmission}[1]{\def\thesheetsubmission{#1}}
\newcommand*{\sheetnumber}[1]{\def\thesheetnumber{#1}}


%----------------------------------------------------------------------------------------

\usepackage{minted}
\usemintedstyle{friendly}
\sheetsubmission{Abgabe bis Di. 29.6.2021 per E-Mail oder Slack}
\sheetnumber{5}
\sheetdate{10.6.2021}
\usepackage{csquotes}
\newcommand{\pybw}[1]{\mintinline[style=bw]{python}{#1}}
\newcommand{\py}[1]{\mintinline{python}{#1}}

\begin{document}

%-------------------------------
%	TITLE SECTION
%-------------------------------

\fancyhead[C]{}
\hrule \medskip % Upper rule
\begin{minipage}[t]{0.295\textwidth}
\raggedright
\footnotesize
Dr. Aaron Kunert \hfill\\   
aaron.kunert@salemkolleg.de \hfill \\
\end{minipage}
\begin{minipage}[t]{0.4\textwidth} 
\centering 
\large 
Einführung in Python\\ 
\normalsize 
Lösung zu Blatt \thesheetnumber{}\\ 
\end{minipage}
\begin{minipage}[t]{0.295\textwidth} 
\raggedleft
\footnotesize
\thesheetdate{}
\hfill\\
\end{minipage}
\medskip\hrule 
\bigskip

%-------------------------------
%	CONTENTS
%-------------------------------

In der First-Class-Cloud findest Du im Ordner \texttt{Quiz} die Datei \texttt{countries.json}. Lade Dir diese Datei in Dein Python-Projekt. 

\section{Datei einlesen}
Verwende das Modul \texttt{json}, um die Datei \texttt{countries.json} (siehe oben) einzulesen, d.h. daraus eine Liste von Ländern (=Dictionaries) zu machen. Gib mittels einer \pybw{for}-Schleife die Hauptstädte aller Länder auf der Konsole aus. Die Ausgabe soll wie folgt aussehen: 
\begin{verbatim}
Kabul ist die Hauptstadt von Afghanistan
Tirana ist die Hauptstadt von Albanien
...
\end{verbatim}


\section{}
Schreibe eine Funktion, die eine beliebige Liste von Ländern (im Format wie in \texttt{countries.json}) annimmt, und eine Liste von Hauptstädten (im Format \pybw{str} ) zurückgibt. 

\section{Länder nach Kontinent filtern}
Schreibe eine Funktion, die eine beliebige Liste von Ländern (im Format wie in \texttt{countries.json}), sowie eine Liste von Kontinenten (im Format \pybw{str}) annimmt. Sie soll dann eine Liste alle Länder zurückgeben, die in einem der angegebenen Kontinente liegen. 

\section{}
Google nach dem Modul \texttt{random}. Schreibe eine Funktion, die eine Liste annimmt. Dabei darf man annehmen, dass die Eingangsliste mindestens 4 Elemente hat und keine doppelten Elemente vorkommen. Die Funktion soll eine Liste mit genau 4 Einträgen zurückgeben, die folgende Eigenschaften hat. 
\begin{itemize}
	\item Der erste Eintrag der Eingangsliste ist in der Ausgangsliste enthalten
	\item Die Position dieses Eintrages ist zufällig (d.h. nicht immer an Position \pybw{0})
	\item Die weiteren Einträge sind zufällig gewählte Elemente der Eingabeliste
	\item Es gibt keine doppelten Einträge in der Ausgangsliste
\end{itemize}
Die Eingangsliste soll dabei nicht verändert werden. 





\end{document}
