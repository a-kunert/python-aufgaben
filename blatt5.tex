\documentclass[a4paper]{article} 
\input{head}
\usepackage{minted}
\usemintedstyle{friendly}
\sheetsubmission{Abgabe bis Di. 29.6.2021 per E-Mail oder Slack}
\sheetnumber{5}
\sheetdate{10.6.2021}
\usepackage{csquotes}
\newcommand{\pybw}[1]{\mintinline[style=bw]{python}{#1}}
\newcommand{\py}[1]{\mintinline{python}{#1}}

\begin{document}

%-------------------------------
%	TITLE SECTION
%-------------------------------

\input{header}

%-------------------------------
%	CONTENTS
%-------------------------------

In der First-Class-Cloud findest Du im Ordner \texttt{Quiz} die Datei \texttt{countries.json}. Lade Dir diese Datei in Dein Python-Projekt. 

\section{Datei einlesen}
Verwende das Modul \texttt{json}, um die Datei \texttt{countries.json} (siehe oben) einzulesen, d.h. daraus eine Liste von Ländern (=Dictionaries) zu machen. Gib mittels einer \pybw{for}-Schleife die Hauptstädte aller Länder auf der Konsole aus. Die Ausgabe soll wie folgt aussehen: 
\begin{verbatim}
Kabul ist die Hauptstadt von Afghanistan
Tirana ist die Hauptstadt von Albanien
...
\end{verbatim}


\section{}
Schreibe eine Funktion, die eine beliebige Liste von Ländern (im Format wie in \texttt{countries.json}) annimmt, und eine Liste von Hauptstädten (im Format \pybw{str} ) zurückgibt. 

\section{Länder nach Kontinent filtern}
Schreibe eine Funktion, die eine beliebige Liste von Ländern (im Format wie in \texttt{countries.json}), sowie eine Liste von Kontinenten (im Format \pybw{str}) annimmt. Sie soll dann eine Liste alle Länder zurückgeben, die in einem der angegebenen Kontinente liegen. 

\section{}
Google nach dem Modul \texttt{random}. Schreibe eine Funktion, die eine Liste annimmt. Dabei darf man annehmen, dass die Eingangsliste mindestens 4 Elemente hat und keine doppelten Elemente vorkommen. Die Funktion soll eine Liste mit genau 4 Einträgen zurückgeben, die folgende Eigenschaften hat. 
\begin{itemize}
	\item Der erste Eintrag der Eingangsliste ist in der Ausgangsliste enthalten
	\item Die Position dieses Eintrages ist zufällig (d.h. nicht immer an Position \pybw{0})
	\item Die weiteren Einträge sind zufällig gewählte Elemente der Eingabeliste
	\item Es gibt keine doppelten Einträge in der Ausgangsliste
\end{itemize}
Die Eingangsliste soll dabei nicht verändert werden. 





\end{document}
