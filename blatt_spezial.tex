\documentclass[a4paper]{article} 
\input{head}

\sheetsubmission{}
\sheetnumber{Spezialaufgaben}
\sheetdate{6.5.2021}


\begin{document}

%-------------------------------
%	TITLE SECTION
%-------------------------------

\input{header}

%-------------------------------
%	CONTENTS
%-------------------------------

\section{GUI}
Ziehe Dir das Repository \texttt{https://github.com/a-kunert/python-gui-minimal-example}. Falls noch nicht geschehen, installiere Dir den Modul \texttt{PySimpleGui}.
Bekomme das Beispiel zum laufen und versuche es nachzuvollziehen. 
Baue noch einen weiteren Button \emph{Reset} ein, der alles wieder auf Anfang setzt. 

\vspace{2pt}

{\footnotesize\textbf{Hinweis:}
	Man kann Module direkt innerhalb bzw. mit Hilfe von Pycharm installieren}


\section{Primfaktorzerlegung}
Lies eine positive ganze Zahl ein und gib die Primfaktorzerlegung dazu aus. Die Eingabe der Zahl 120 sollte dann etwa folgende Ausgabe produzieren:

\vspace{2pt}
\texttt{2,2,2,3,5}.

\vspace{2pt}

für größere Zahlen bietet sich auch eine Darstellung in der Form 

\vspace{2pt}

\texttt{2\^{}3,3,5}

\vspace{2pt}
an. 


\section{Primfaktorzerlegung + GUI}
Baue eine GUI mit einem Eingabefeld (positive, ganze Zahl) auf der Ausgabenseite soll die Primfaktorzerlegung angegeben werden. Zusätzlich soll die benötigte Zeit angezeigt werden, die die Faktorzerlegung benötigt hat. 


Kannst Du Deinen Faktorisierungs-Algorithmus noch optimieren? 


\section{}
Erweitere das Programm aus Aufgabe 3. Messe nun zusätzlich die Zeit, die Python benötigt, um alle Primfaktoren wieder zusammenzumultiplizieren (was die ursprüngliche Zahl ergibt). 
Zeige die beiden Zeiten an und gib zusätzlich an, um welchen Faktor die beiden Zeiten voneinander abweichen. 





\end{document}
