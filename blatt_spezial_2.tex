\documentclass[a4paper]{article} 
\input{head}

\sheetsubmission{}
\sheetnumber{Spezialaufgaben II}
\sheetdate{20.5.2021}
\usepackage{minted}
\usemintedstyle{friendly}
\usepackage{csquotes}
\newcommand{\pybw}[1]{\mintinline[style=bw]{python}{#1}}
\newcommand{\py}[1]{\mintinline{python}{#1}}



\begin{document}

%-------------------------------
%	TITLE SECTION
%-------------------------------

\input{header}

%-------------------------------
%	CONTENTS
%-------------------------------

\section{Rekursive Funktionen}
Die Folge der Fibonacci-Zahlen ist wie folgt definiert: Das $n$-te Folgenglied ist die Summe der beiden vorherigen Folgenglieder. Die ersten beiden Folgenglieder sind als $1$ definiert. Die ersten Fibonacci-Zahlen sind $1,1,2,3,5,8,13,\dots$. 

\vspace{2pt}

Schreibe eine Funktion, die eine Zahl $n$ erwartet und die $n$-te Fibonacci-Zahl zurückgibt. Realisiere die Funktion einmal als \emph{rekursive} Funktion und einmal über eine Schleife. Miss jeweils die Ausführungszeiten. Welche ist effizienter und warum? 

\section{Webseiten verstehen}
Schaue Dir eine Webseite mit den Entwicklerwerkzeugen an. Diese kannst Du in Chrome/Firefox mit der Tastaturkombination  Ctrl/Cmd + Shift + I öffnen. Welche Struktur hat so eine Seite? 
Verwende das Modul \pybw{requests} um den Quellcode einer beliebige Webseite in ein Python-Programm zu bringen.  


\section{Webscraping}
Im Folgenden geht es darum, eine Liste von Ländern samt Daten von der Webseite 

\texttt{https://de.wikipedia.org/wiki/Liste\_der\_Staaten\_der\_Erde} auszulesen und als JSON-Datei abzuspeichern. Du kannst Dich grob an der Anleitung auf \texttt{https://realpython.com/beautiful-soup-web-scraper-python} orientieren. 
Die Einträge des JSONS sollen am Ende wie folgt aussehen: 

\begin{minted}[style=bw]{python}
{
 "name":"Afghanistan"
 "capital":"Kabul"
 "population":35500000
 "iso":"AFG"
}
\end{minted}

Gehe dabei nach folgenden Schritten vor: 

\begin{itemize}
	\item Öffne die Seite mittels \pybw{requests}.
	\item Finde die Tabelle, indem Du mittels \pybw{beautifulsoup4} den Quellcode der Seite filterst.
	\item Mittels einer Schleife über die Zeilen und Spalten der Tabelle kannst Du die Daten in ein Listen/Dictionary-Format bringen. 
	\item Verwende reguläre Ausdrücke (das Modul \pybw{re}) um die Fußnotenverweise (z.B. [7] bei Abchasien) zu entfernen. 
	\item Analog lassen sich die Einwohnerzahlen bereinigen
	\item Wandle die Einwohnezahlen zu \pybw{integers} um. 
	\item Schaffst Du es die Anmerkungen zum Ländernamen (wie z.B. bei Dänemark und Frankreich) ebenso verschwinden zu lassen? 
	\item Speichere mittels \pybw{json} die Daten in einer Datei ab. 
\end{itemize}







\end{document}
