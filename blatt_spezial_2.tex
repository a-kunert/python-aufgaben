\documentclass[a4paper]{article} 
\addtolength{\hoffset}{-2.25cm}
\addtolength{\textwidth}{4.5cm}
\addtolength{\voffset}{-3.25cm}
\addtolength{\textheight}{5cm}
\setlength{\parskip}{0pt}
\setlength{\parindent}{0in}

%----------------------------------------------------------------------------------------
%	PACKAGES AND OTHER DOCUMENT CONFIGURATIONS
%----------------------------------------------------------------------------------------

\usepackage{blindtext} % Package to generate dummy text
\usepackage{charter} % Use the Charter font
%\usepackage{lmodern} % Use the Charter font
\usepackage[utf8]{inputenc} % Use UTF-8 encoding
\usepackage{microtype} % Slightly tweak font spacing for aesthetics
\usepackage[english, ngerman]{babel} % Language hyphenation and typographical rules
\usepackage{amsthm, amsmath, amssymb} % Mathematical typesetting
\usepackage{float} % Improved interface for floating objects
\usepackage[final, colorlinks = true, 
            linkcolor = black, 
            citecolor = black]{hyperref} % For hyperlinks in the PDF
\usepackage{graphicx, multicol} % Enhanced support for graphics
\usepackage{xcolor} % Driver-independent color extensions
\usepackage{marvosym, wasysym} % More symbols
\usepackage{rotating} % Rotation tools
\usepackage{censor} % Facilities for controlling restricted text
\usepackage{listings, style/lstlisting} % Environment for non-formatted code, !uses style file!
\usepackage{pseudocode} % Environment for specifying algorithms in a natural way
\usepackage{style/avm} % Environment for f-structures, !uses style file!
\usepackage{booktabs} % Enhances quality of tables
\usepackage{tikz-qtree} % Easy tree drawing tool
\usepackage{ifthen}
\usepackage{lastpage}
\usepackage{titlesec}
\tikzset{every tree node/.style={align=center,anchor=north},
         level distance=2cm} % Configuration for q-trees
\usepackage{style/btree} % Configuration for b-trees and b+-trees, !uses style file!
\usepackage[backend=biber,style=numeric,
            sorting=nyt]{biblatex} % Complete reimplementation of bibliographic facilities
\addbibresource{ecl.bib}
\usepackage{csquotes} % Context sensitive quotation facilities
\usepackage{fancyhdr} % Headers and footers
\pagestyle{fancy} % All pages have headers and footers
\fancyhead{}\renewcommand{\headrulewidth}{0pt} % Blank out the default header
\fancyfoot[L]{\thesheetsubmission{}} % Custom footer text
\fancyfoot[C]{} % Custom footer text
\fancyfoot[R]{\ifthenelse{\pageref{LastPage} > 1}{\footnotesize Seite \thepage{} von \pageref{LastPage} }} % Custom footer text
\newcommand{\note}[1]{\marginpar{\scriptsize \textcolor{red}{#1}}} % Enables comments in red on margin

\titleformat{\section}
{\normalfont\Large\bfseries}{Aufgabe~\thesection}{1em}{\normalsize}
\titlespacing*{\section}{0em}{6ex}{2ex}


\renewcommand{\theenumi}{\alph{enumi}}
\renewcommand\labelenumi{(\theenumi)}

\newcommand*{\sheetdate}[1]{\def\thesheetdate{#1}}
\newcommand*{\sheetsubmission}[1]{\def\thesheetsubmission{#1}}
\newcommand*{\sheetnumber}[1]{\def\thesheetnumber{#1}}


%----------------------------------------------------------------------------------------


\sheetsubmission{}
\sheetnumber{Spezialaufgaben II}
\sheetdate{20.5.2021}
\usepackage{minted}
\usemintedstyle{friendly}
\usepackage{csquotes}
\newcommand{\pybw}[1]{\mintinline[style=bw]{python}{#1}}
\newcommand{\py}[1]{\mintinline{python}{#1}}



\begin{document}

%-------------------------------
%	TITLE SECTION
%-------------------------------

\fancyhead[C]{}
\hrule \medskip % Upper rule
\begin{minipage}[t]{0.295\textwidth}
\raggedright
\footnotesize
Dr. Aaron Kunert \hfill\\   
aaron.kunert@salemkolleg.de \hfill \\
\end{minipage}
\begin{minipage}[t]{0.4\textwidth} 
\centering 
\large 
Einführung in Python\\ 
\normalsize 
Lösung zu Blatt \thesheetnumber{}\\ 
\end{minipage}
\begin{minipage}[t]{0.295\textwidth} 
\raggedleft
\footnotesize
\thesheetdate{}
\hfill\\
\end{minipage}
\medskip\hrule 
\bigskip

%-------------------------------
%	CONTENTS
%-------------------------------

\section{Rekursive Funktionen}
Die Folge der Fibonacci-Zahlen ist wie folgt definiert: Das $n$-te Folgenglied ist die Summe der beiden vorherigen Folgenglieder. Die ersten beiden Folgenglieder sind als $1$ definiert. Die ersten Fibonacci-Zahlen sind $1,1,2,3,5,8,13,\dots$. 

\vspace{2pt}

Schreibe eine Funktion, die eine Zahl $n$ erwartet und die $n$-te Fibonacci-Zahl zurückgibt. Realisiere die Funktion einmal als \emph{rekursive} Funktion und einmal über eine Schleife. Miss jeweils die Ausführungszeiten. Welche ist effizienter und warum? 

\section{Webseiten verstehen}
Schaue Dir eine Webseite mit den Entwicklerwerkzeugen an. Diese kannst Du in Chrome/Firefox mit der Tastaturkombination  Ctrl/Cmd + Shift + I öffnen. Welche Struktur hat so eine Seite? 
Verwende das Modul \pybw{requests} um den Quellcode einer beliebige Webseite in ein Python-Programm zu bringen.  


\section{Webscraping}
Im Folgenden geht es darum, eine Liste von Ländern samt Daten von der Webseite 

\texttt{https://de.wikipedia.org/wiki/Liste\_der\_Staaten\_der\_Erde} auszulesen und als JSON-Datei abzuspeichern. Du kannst Dich grob an der Anleitung auf \texttt{https://realpython.com/beautiful-soup-web-scraper-python} orientieren. 
Die Einträge des JSONS sollen am Ende wie folgt aussehen: 

\begin{minted}[style=bw]{python}
{
 "name":"Afghanistan"
 "capital":"Kabul"
 "population":35500000
 "iso":"AFG"
}
\end{minted}

Gehe dabei nach folgenden Schritten vor: 

\begin{itemize}
	\item Öffne die Seite mittels \pybw{requests}.
	\item Finde die Tabelle, indem Du mittels \pybw{beautifulsoup4} den Quellcode der Seite filterst.
	\item Mittels einer Schleife über die Zeilen und Spalten der Tabelle kannst Du die Daten in ein Listen/Dictionary-Format bringen. 
	\item Verwende reguläre Ausdrücke (das Modul \pybw{re}) um die Fußnotenverweise (z.B. [7] bei Abchasien) zu entfernen. 
	\item Analog lassen sich die Einwohnerzahlen bereinigen
	\item Wandle die Einwohnezahlen zu \pybw{integers} um. 
	\item Schaffst Du es die Anmerkungen zum Ländernamen (wie z.B. bei Dänemark und Frankreich) ebenso verschwinden zu lassen? 
	\item Speichere mittels \pybw{json} die Daten in einer Datei ab. 
\end{itemize}







\end{document}
